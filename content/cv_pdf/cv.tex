\nonstopmode
%%%%%%%%%%%%%%%%%%%%%%%%%%%%%%%%%%%%%%%%%
% "ModernCV" CV and Cover Letter
% LaTeX Template
% Version 1.11 (19/6/14)
%
% This template has been downloaded from:
% http://www.LaTeXTemplates.com
%
% Original author:
% Xavier Danaux (xdanaux@gmail.com)
%
% License:
% CC BY-NC-SA 3.0 (http://creativecommons.org/licenses/by-nc-sa/3.0/)
%
% Important note:
% This template requires the moderncv.cls and .sty files to be in the same
% directory as this .tex file. These files provide the resume style and themes
% used for structuring the document.
%
%%%%%%%%%%%%%%%%%%%%%%%%%%%%%%%%%%%%%%%%%

%----------------------------------------------------------------------------------------
%	PACKAGES AND OTHER DOCUMENT CONFIGURATIONS
%----------------------------------------------------------------------------------------

\documentclass[11pt,a4paper,serif,linkcolor=true]{moderncv} % Font sizes: 10, 11, or 12; paper sizes: a4paper, letterpaper, a5paper, legalpaper, executivepaper or landscape; font families: sans or roman

\usepackage[utf8x]{inputenc}
\moderncvstyle{casual}
\moderncvcolor{grey}
\usepackage[scale=0.8]{geometry} % Reduce document margins

\setlength{\hintscolumnwidth}{3cm} % Uncomment to change the width of the dates column
%\setlength{\makecvtitlenamewidth}{10cm} % For the 'classic' style, uncomment to adjust the width of the space allocated to your name

% own customizations
\usepackage[ngerman,english]{babel}
%\usepackage[default,osfigures,scale=0.95]{opensans}
\usepackage[scaled]{helvet}
\usepackage{fixltx2e}
\usepackage{eurosym}
\usepackage{enumitem}
\setenumerate{label=(\roman*),itemsep=3pt,topsep=3pt}

\renewcommand*\familydefault{\sfdefault} %% Only if the base font of the document is to be sans serif
\usepackage[T1]{fontenc}


\definecolor{linkblue}{RGB}{50,107,164} % Bootstrap style
\usepackage{url}
\urlstyle{same}
\AtBeginDocument{\hypersetup{colorlinks=true,urlcolor=linkblue,linkcolor=gray}}

%----------------------------------------------------------------------------------------
%	NAME AND CONTACT INFORMATION SECTION
%----------------------------------------------------------------------------------------

\firstname{Florian M.}
\familyname{Wagner}
\title{Curriculum vitae}
\email{mail@fwagner.info}
\homepage{www.fwagner.info}{www.fwagner.info}
\photo[150pt][0pt]{../static/fwagner2.jpg}
\extrainfo{Last updated on \today}

%----------------------------------------------------------------------------------------

\begin{document}

\makecvtitle % Print the CV title

\section{Professional Experience}
\cventry{since Oct. 2019}{Substitute professor of Applied Geophysics}{RWTH Aachen University}{Aachen}{Germany}{}
\cventry{Sept. 2018\\-- Dec. 2018}{Visiting scholar}{Earth \& Environmental Sciences Area, Lawrence Berkeley National Laboratory}{Berkeley, CA}{USA}{Joint inversion of seismic refraction and electrical resistivity data.}
\cventry{Aug. 2016\\-- Sept. 2019}{Postdoctoral research associate}{University of Bonn, Geophysics Section}{Bonn}{Germany}{Research on joint and process-based imaging for permafrost characterization and monitoring as well as teaching \& theses supervision on Bachelor and Master level.}
\cventry{Nov. 2011\\-- May 2016}{Research associate}{GFZ German Research Centre for Geosciences, Section 6.3 - Geological Storage}{Potsdam}{Germany}{Research on geoelectrical CO$\textsubscript{2}$ monitoring, field experiments, server administration.}

\section{Education}
\cventry{2012--2016}{Doctor of Sciences}{ETH Zurich}{Department of Earth Sciences}{}{Supervisors: Prof. Dr. Hansruedi Maurer and Dr. Cornelia Schmidt-Hattenberger (GFZ)\\Dissertation: \url{www.diss.fwagner.info}}
\cventry{2009 --2011}{Master of Science}{IDEA League: TU Delft, ETH Zurich, RWTH Aachen}{}{}{Joint M.Sc. in Applied Geophysics (\url{http://www.idealeague.org/geophysics})}
\cventry{2006--2009}{Bachelor of Science}{RWTH Aachen University}{}{}{B.Sc. in Georesources Management}

\section{Scientific interests}
\vspace{-.4cm}
\cvline{}{
\begin{itemize}[noitemsep,topsep=0pt]
\item Geophysical monitoring of subsurface fluid migration
\item Permafrost characterization and monitoring
\item Geothermal and geological storage reservoirs
\item Tomographic experimental design
\item Numerical modeling and (process-based/joint) inversion
\item Scientific software development (\url{www.pygimli.org})
\end{itemize}
}
\vspace{-.4cm}

%\section{Practical Experience}
%\cventry{March 2009}{Internship}{Trasswerke Meurin}{Andernach}{Germany}{Internship at Trasswerke Meurin Produktions- und Handelsgesellschaft mbH (\url{www.meurin.com}). Gained experience in the excavation of volcanic rocks for the production of quality construction materials including sieve analyses and strength tests.}
%\cventry{June 2010}{Field Campaign}{ETH Zurich}{Kloten and Laegeren}{Switzerland}{Comprehensive geophysical field work incorporating data acquisition, processing and reporting. Applied measuring techniques included: Electrical Resistivity Tomography (ERT), Seismic Refraction Tomography (SRT), Ground Penetrating Radar (GPR), Electromagnetics (EM31 and EM38), Transient Electromagnetics (TEM) and Magnetics.}
%\cventry{July 2010}{Internship}{DMT GmbH \& Co. KG}{Essen}{Germany}{Large-scale 3D seismic survey by DMT GmbH \& Co. KG (\url{www.dmt.de/en/home.html}) in Jointville, France.
%\newline
%\newline
%Gained hands-on experiences in:
%\begin{itemize}
%\item Large-scale data acquisition
%\item Reflection \& Refraction Seismics
%\item Vertical Seismic Profiling (VSP)
%\item Well Logging
%\item Quality Control
%\end{itemize}
%}
%\cventry{September 2013}{Research visit}{University of Alberta}{Edmonton}{Canada}{
%I worked on acoustic and electrical analysis of reservoir sandstones in the Geomechanical Reservoir Experimental Facility (\url{www.geo-ref.ca}) of the University of Alberta in Edmonton, Canada, under supervision of Dr. Chalaturnyk and his team.
%}

\section{Teaching}
%\setlength{\hintscolumnwidth}{4cm}
\cvline{since 2019\\RWTH Aachen University}{\vspace{-\baselineskip}
\begin{itemize}[noitemsep,topsep=0pt]
\item \textit{Physics of the Earth}, Lecture, B.Sc.
\item \textit{Geothermics}, Lecture and exercise, M.Sc.
\item \textit{Application of geophysical prospection methods in Earth and Environmental Sciences}, Lecture and exercise, M.Sc.
\item \textit{Theory of geophysical prospection methods}, Lecture and exercise, M.Sc.
\end{itemize}
}\vspace{-.35cm}
\cvline{2016 -- 2019\\University of Bonn}{\vspace{-\baselineskip}
\begin{itemize}[noitemsep,topsep=0pt]
\item \textit{Environmental geophysics}, Lecture, B.Sc. % & Universität Bonn & B.Sc. & DE & SS17\\
\item \textit{Inverse modeling}, Lecture and exercise, M.Sc. % & Universität Bonn & M.Sc. & EN & SS17\\
\item \textit{Hydrogeophysical process simulation}, Block course, M.Sc. % & Universität Bonn & M.Sc. & DE / EN & WS16/17 \& WS17/18\\
\item \textit{Applied hydrogeophysics}, Seminar \& field course, M.Sc. % & Universität Bonn & M.Sc. & DE & SS18 \\
\item Supervision of five Bachelor and five Master theses
\end{itemize}
}\vspace{-.35cm}
\cvline{2013 -- 2015\\ETH Zurich}{\vspace{-\baselineskip}
\begin{itemize}[noitemsep,topsep=0pt]
\item \textit{Geophysical field work and data processing}, Field course \& block course, M.Sc. % & ETH Zürich & M.Sc. & EN & SS13, SS14, SS15\\
\item Supervision of a master thesis and committee member during oral defense
\end{itemize}
} \vspace{-.35cm}
\cvline{Other}{\vspace{-\baselineskip}
\begin{itemize}[noitemsep,topsep=0pt]
\item \textit{Modeling \& inversion with \href{http://www.pygimli.org/}{pyGIMLi}}, One-day course, Univ. of Leipzig, Sept. 2016 and Jan. 2019
\item \textit{Scientific computing with Python}, One-day course, GFZ Potsdam, Nov. 2012
\item \textit{Best Practices for Modern Open-Source Research Codes}, Workshop at AGU Fall Meeting, Washington D.C., Dec. 2018
\end{itemize}
}
\vspace{-.4cm}

\section{Scholarships \& Awards}
\cvline{Aug. 2021}{Best teaching award by the student council Earth sciences and resource management at RWTH Aachen University (\href{https://www.gge.eonerc.rwth-aachen.de/cms/E-ON-ERC-GGE/Das-Institut/Aktuelle-Meldungen-Institut/~qoeyz/Lehrpreis-Wagner/?lidx=1}{\nolinkurl{https://www.gge.eonerc.rwth-aachen.de/cms/E-ON-ERC-GGE/Das-Institut/Aktuelle-Meldungen-Institut/~qoeyz/Lehrpreis-Wagner/?lidx=1}})}
\cvline{Jan. 2021}{Teaching prize awarded by the Faculty of Georesources and Materials Engineering (\href{https://www.fgeo.rwth-aachen.de/cms/Geowissenschaften-und-Geographie/Die-Fachgruppe/Aktuell/Meldungen/~mfudr/Lehrpreis-der-Fakultaet/?lidx=1}{\nolinkurl{https://www.fgeo.rwth-aachen.de/cms/Geowissenschaften-und-Geographie/Die-Fachgruppe/Aktuell/Meldungen/~mfudr/Lehrpreis-der-Fakultaet/?lidx=1}})}
\cvline{Sept. 2017}{\textit{Wasser-Monitoring-Preis} (50.000 \euro) awarded by the Dr. Erich Ritter foundation in cooperation with the Water Science Alliance e.V.  (\href{http://www.deutsches-stiftungszentrum.de/aktuelles/2017_09_12_wasser-monitoring-preis}{\nolinkurl{www.deutsches-stiftungszentrum.de/aktuelles/2017_09_12_wasser-monitoring-preis}})}
\cvline{Apr. 2016}{1\textsuperscript{st} place in category "EGU Talk" at the 11\textsuperscript{th} annual GFZ PhD Day, Potsdam, Germany}
\cvline{Sept. 2014}{Best oral presentation award at the 4\textsuperscript{th} Science Forum of the Helmholtz-Alberta-Initiative (\href{http://www.helmholtz-alberta.org}{www.helmholtz-alberta.org}), Edmonton, Canada}
\cvline{Sept. 2013}{Best oral presentation award at the 3\textsuperscript{rd} Science Forum of the Helmholtz-Alberta-Initiative, Edmonton, Canada}
\cvline{Sept. 2012}{Best oral presentation award at the 2\textsuperscript{nd} Science Forum of the Helmholtz-Alberta-Initiative, Potsdam, Germany}
\cvline{2009--2011}{Scholarship from the education fund of North Rhine-Westphalia}
\cvline{2009--2011}{Private scholarship from DEA Deutsche Erdoel AG (\href{http://www.dea-group.com/en}{\nolinkurl{www.dea-group.com/en}})}

\section{Professional services}
\cvline{Chair and convener}{Chair at the EGU General Assembly for \href{https://meetingorganizer.copernicus.org/EGU21/session/39299}{the session on near-surface geophysical imaging} (since 2019) and convener at AGU 2018 for \href{https://agu.confex.com/agu/fm18/prelim.cgi/Session/56864}{session on geoscientific open-source software}}
\cvline{Reviewer for}{\textit{Advances in Geosciences, Computers and Geosciences, Frontiers in Earth Sciences, Geophysics, Geophysical Journal International, International Journal of Greenhouse Gas Control, Journal of Applied Geophysics, Journal of Geophysical Research - Earth Surface, Near Surface Geophysics, Pure and Applied Geophysics, Solid Earth, The Cryosphere, Vadose Zone Journal, Water Resources Research}}
\cvline{Committees \& thesis supervision}{Member in 5 Ph.D. committees (University of Fribourg, University of Li\`{e}ge, RWTH Aachen University) and supervision of > 20 B.Sc. and M.Sc. theses}
\cvline{Software development and support}{\textit{pyGIMLi}: An open-source library for modeling and inversion in geophysics (\url{www.pygimli.org})}

\section{Invited contributions at international conferences}
%\footnotesize
\begin{itemize}[label={},itemindent=-1.8em,leftmargin=1.8em]
	\item \textbf{Wagner, F. M.}, Rücker, C., Günther, T., Dinsel, F., Skibbe, N., Weigand, M., Hase, J. (2020): Open-source hydrogeophysical modeling and inversion with \href{http://www.pygimli.org/}{pyGIMLi} 1.1: Recent advances and examples in research and education. \textbf{EGU} General Assembly 2020, Vienna,
	4-8 May 2020,
	\href{https://doi.org/10.5194/egusphere-egu2020-18751}{\nolinkurl{DOI:10.5194/egusphere-egu2020-18751}}.
	\item \textbf{Wagner, F. M.}, Rücker, C., Günther, T. (2017): Reproducible
	hydrogeophysical inversions through the open-source library
	\href{http://www.pygimli.org/}{pyGIMLi}. \textbf{AGU} Fall Meeting, New Orleans,
	11-15 Dec 2017, Open-Source Software in the Geosciences (NS41B-0016),
	\href{https://doi.org/10.5281/zenodo.1095621}{\nolinkurl{DOI:10.5281/zenodo.1095621}}.
	\item \textbf{Wagner, F. M.}, Wiese, B., Schmidt-Hattenberger, C., Maurer, H.
	(2016): Insights on CO\textsubscript{2} Migration Based on a
	Multi-physical Inverse Reservoir Modeling Framework.
	78\textsuperscript{th} \textbf{EAGE} Conference \& Exhibition, 30 May - 2 June
	2016, Vienna, WS10-Quantitative Data Integration and Joint Inversion
	from Surface to Reservoir,
	\href{https://doi.org/10.3997/2214-4609.201601659}{\nolinkurl{DOI:10.3997/2214-4609.201601659}}.
\end{itemize}

\input{publications}

\end{document}
